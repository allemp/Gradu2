\chapter{Methods}

%--------------------------------------------------------------------------------------------------------------------------------------------
\section{Research setup}
\subsection{Scope}
The scope of the study is limited to 5 performance metrics of 3 different ML models trained and tested on 3 different datasets using a distributed computing framework Ray Tune \parencite{liawTuneResearchPlatform2018}.

TODO different models, different datasets

TODO Vertailukriteeristö: tapana ohjelmistopuolella + tapana koneoppimispuolella

TODO different resources (memory, time, accuracy)



\subsection{Research Questions}
This master's thesis asks the following research questions:
\begin{itemize}
    %\item \emph{RQ1}: How can we measure the performance of real-world ML systems using Ray Tune \parencite{liawTuneResearchPlatform2018}?
    \item \emph{RQ1}: How early stopping affects performance metrics during hyperparameter optimization?
    \item \emph{RQ2}: How early stopping affects performance metrics during neural architecture optimization?
\end{itemize}

\subsection{Methodology}

Methodology used is expanded from an existing methodology for machine learning experiment design \parencite{fernandez-lozanoMethodologyDesignExperiments2016} to include AutoML and 

%--------------------------------------------------------------------------------------------------------------------------------------------
\section{Experiments}
%\subsection{Overview}
\subsection{Datasets}
% Tarkemmin esimerkkinä, loput vähemmän tarkasti
MNIST \parencite{dengMNISTDatabaseHandwritten2012}

% Satunnaisesti valitaan X kappaletta
Penn Machine Learning Benchmarks \parencite{olsonPMLBLargeBenchmark2017}




\subsection{Machine Learning algorithms}
\subsection{Metrics}
\subsection{Training and validation}
\subsection{Inference}

%--------------------------------------------------------------------------------------------------------------------------------------------
