\chapter{Discussion}
\label{chap:discuss}
\section{Research Questions revisited}
%\item \emph{RQ3}: Can resource usage metrics from early stopped models be used to predict resource usage metrics of fully trained models?
%\item \emph{RQ4}: Can resource usage metrics from models trained on reduced training datasets be used to predict resource usage metrics of models trained on full training datasets?
%\item \emph{RQ5}: Can application performance metrics from early stopped models be used to predict application performance metrics of fully trained models?
%\item \emph{RQ6}: Can application performance metrics from models trained on reduced training datasets be used to predict application performance metrics of models trained on full training datasets?
\subsection{Research question RQ1}
\subsection{Research question RQ2}
\subsection{Research question RQ3}

\section{Interpretation}
\subsection{Implications for research}
\subsection{Implications for practice}

\section{Limitations}
% Internal validity, external validity jätetty pois
\subsection{Datasets}
\subsection{Machine Learning algorithms}
\subsection{Metrics}
\subsection{Training and validation}
\subsection{Inference}


\section{Related Work}


Cardoso Silva et al. \parencite*{cardososilvaBenchmarkingMachineLearning2020} in their paper identify key system metrics for monitoring a production machine learning system. Key metrics identified include task completion time, CPU and GPU usage, memory usage, disk input/output and network traffic and their collection was implemented in a tool called \emph{Ubenchmark} \parencite{cardososilvaBenchmarkingMachineLearning2020}. The researchers in particular focus on empirically monitoring and performance benchmarking of the machine learning system.


\section{Future Work}
TODO This is a discussion chapter