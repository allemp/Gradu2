\chapter{Machine Learning Operations}

% TODO Tarvitseeko luku omaa pientä introa?
% Luvun scope
This chapter introduces the basic concepts of ML in \autoref{sec:ml} and basic concepts of DevOps in \autoref{sec:devops}. Finally in \autoref{sec:mldevops} DevOps and ML are combined to form a new concept of MLOps.

%--------------------------------------------------------------------------------------------------------------------------------------------
\section{Introduction Machine Learning}
\label{sec:ml}

\subsection{Overview}
TODO what is ML

TODO supervised vs unsupervised vs reinforcement

Unsupervised learning has the advantage of not requiring labeled data which is an advantage for problems where labels are uncommon \parencite{leBuildingHighlevelFeatures2012}. Unsupervised pretraining or feature learning on large datasets can yield to surprising results such as human facial recognition \parencite{leBuildingHighlevelFeatures2012}. Unsupervised machine learning techniques can be used as part of a learning pipeline as a preprocessing or postprocessing step even if the rest of the system uses supervised machine learning.

TODO training vs serving

\subsection{Machine learning algorithms}

TODO Model evaluation


\subsection{Hyperparameter optimization}

Parameters given as part of a configuration to the machine learning model are called hyperparameters \parencite{yangHyperparameterOptimizationMachine2020}. Hyperparameter selection is a difficult and automatic tuning of hyperparameters can help achieve state-of-the-art performance \parencite{maclaurinGradientbasedHyperparameterOptimization2015}. Hyperparameter optimization techniques include grid search, random search, gradient based optimization and Bayesian optimization and they have different benefits and limitations \parencite{yangHyperparameterOptimizationMachine2020}. 

Neural Architecture optimization and Meta modeling are similar to hyperparameter optimization where model structure or modeling algorithm is treated as a tunable parameter \parencite{bakerAcceleratingNeuralArchitecture2017}. Traditional hyperparameter tuning methods such as Bayesian optimization are unfeasible for more than 10-20 hyperparameters \parencite{maclaurinGradientbasedHyperparameterOptimization2015}.

Performance prediction is an important step to reduce the amount of computation required for neural architecture search and hyperparameter optimization \parencite{bakerAcceleratingNeuralArchitecture2017}. TODO early stopping reference. 



%--------------------------------------------------------------------------------------------------------------------------------------------
\section{DevOps}
\label{sec:devops}

\subsection{Overview}

There is little consensus on the exact definition of DevOps, but especially collaboration between development and operation is emphasized \parencite{mishraDevOpsSoftwareQuality2020,wallerIncludingPerformanceBenchmarks2015}. DevOps can be studied from different points of view such as culture, collaboration, automation, measurements and monitoring \parencite{mishraDevOpsSoftwareQuality2020, wallerIncludingPerformanceBenchmarks2015}. This thesis is mostly focused on the automation, measurements and monitoring parts of DevOps.


TODO: picture about devops

Continuous integration, continuous deployment and continuous monitoring are well known practices in DevOps \parencite{wallerIncludingPerformanceBenchmarks2015} describing the automatic nature of integrating, deploying and monitoring code changes. Performance profiling and monitoring are similar activities and the main difference is whether it's done during the development process or during operations respectively \parencite{wallerIncludingPerformanceBenchmarks2015}. DevOps bridges the gap between evaluating performance during the development process and during operations \parencite{brunnertPerformanceorientedDevOpsResearch2015}.


TODO resource allocation/resource consumption, small memory software, benchmarking

\subsection{Performance metrics}

Performance metrics are fundamental to all activities involving performance evaluation such as profiling or monitoring \parencite{brunnertPerformanceorientedDevOpsResearch2015}. Common metrics involve measuring the CPU, but other metrics such as memory usage, network traffic or I/O usage are not as well defined as a CPU metric \parencite{brunnertPerformanceorientedDevOpsResearch2015}. 

\begin{itemize}
    \item Task Completion time
    \item Throughput
    \item Latency
    \item CPU usage
    \item GPU usage
    \item RAM usage
    \item VRAM usage
    \item I/O usage
    \item Network traffic
\end{itemize}

\subsection{Performance tuning}

Preprocessing

Training

Serving Latency

Resource demands might change depending on the inputs \parencite{brunnertPerformanceorientedDevOpsResearch2015} making it important to systematically measure performance not only based on code changes but also on configuration changes or even data changes.

%--------------------------------------------------------------------------------------------------------------------------------------------
\section{MLOps}
\label{sec:mldevops}

\subsection{Overview}

Requirements for a machine learning system are different depending on the task. For example speech and object recognition might have no particular performance requirements during training but strict latency and computational resource restrictions when deployed to serve large amounts users \parencite{hintonDistillingKnowledgeNeural2015}. One of the key areas of MLOps is using machine learning in production systems in addition to data processing and machine learning model training.

Performance measuring software is not new, but ML brings additional challenges in the form of models and data which requires a modified approach \parencite{breckMLTestScore2017a}. It is also important to note, that not every data scientist or machine learning engineer working on machine learning systems has a software engineering background \parencite{finzerDataScienceEducation2013} and might lack the necessary knowledge to apply software engineering best practices to machine learning systems.

TODO what DevOps brings to ML

TODO Continuous Training

\subsection{AutoML}

Machine learning systems in addition to machine learning performance metrics and system performance metrics will have their performance metrics tied to product or organization metrics such as user churn rate or click-through rate \parencite{shankarOperationalizingMachineLearning2022}. Choosing the right metrics to evaluate a machine learning system is important and the metrics will be different for different machine learning systems \parencite{shankarOperationalizingMachineLearning2022}.

Automated Machine Learning (AutoML) aims to minimize human intervention in completing data analytics tasks using machine learning algorithms \parencite{yangIoTDataAnalytics2022}.

\subsection{Performance prediction and early stopping}

