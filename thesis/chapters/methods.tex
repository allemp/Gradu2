\chapter{Methods}
\label{chap:methods}

%--------------------------------------------------------------------------------------------------------------------------------------------
\section{Research setup}
\subsection{Scope}

In this thesis the choice of machine learning algorithms is limited to implementations with iterative training and a possibility for metric collection between training steps. 

The scope of the study is limited to 5 performance metrics of 3 different ML models trained and tested on 3 different datasets using a distributed computing framework Ray Tune \parencite{liawTuneResearchPlatform2018}.

TODO different models, different datasets

TODO Vertailukriteeristö: tapana ohjelmistopuolella + tapana koneoppimispuolella

TODO different resources (memory, time, accuracy)

%\begin{itemize}
%    \item Task Completion time
%    \item Throughput
%    \item Latency
%    \item CPU usage
%    \item GPU usage
%    \item RAM usage
%    \item VRAM usage
%    \item I/O usage
%    \item Network traffic
%\end{itemize}


\subsection{Research Questions}
This master's thesis asks the following research questions:
\begin{itemize}
    \item \emph{RQ1}: How do changes in hyperparameters affect system performance during model training?
    \item \emph{RQ2}: How does early stopping on system performance criteria affect computational budgets during model training?
          
\end{itemize}

\subsection{Methodology}

Methodology used is expanded from an existing methodology for machine learning experiment design \parencite{fernandez-lozanoMethodologyDesignExperiments2016} to include AutoML and

%--------------------------------------------------------------------------------------------------------------------------------------------
\section{Experiments}
\label{sec:experiments}
%\subsection{Overview}
\subsection{Datasets}
% Tarkemmin esimerkkinä, loput vähemmän tarkasti
MNIST \parencite{dengMNISTDatabaseHandwritten2012}

% Satunnaisesti valitaan X kappaletta
Penn Machine Learning Benchmarks \parencite{olsonPMLBLargeBenchmark2017}




\subsection{Machine Learning algorithms}
\subsection{Metrics}
\subsection{Training and validation}
\subsection{Inference}

%--------------------------------------------------------------------------------------------------------------------------------------------
